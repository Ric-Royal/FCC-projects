% Options for packages loaded elsewhere
\PassOptionsToPackage{unicode}{hyperref}
\PassOptionsToPackage{hyphens}{url}
%
\documentclass[
]{article}
\usepackage{amsmath,amssymb}
\usepackage{lmodern}
\usepackage{iftex}
\ifPDFTeX
  \usepackage[T1]{fontenc}
  \usepackage[utf8]{inputenc}
  \usepackage{textcomp} % provide euro and other symbols
\else % if luatex or xetex
  \usepackage{unicode-math}
  \defaultfontfeatures{Scale=MatchLowercase}
  \defaultfontfeatures[\rmfamily]{Ligatures=TeX,Scale=1}
\fi
% Use upquote if available, for straight quotes in verbatim environments
\IfFileExists{upquote.sty}{\usepackage{upquote}}{}
\IfFileExists{microtype.sty}{% use microtype if available
  \usepackage[]{microtype}
  \UseMicrotypeSet[protrusion]{basicmath} % disable protrusion for tt fonts
}{}
\makeatletter
\@ifundefined{KOMAClassName}{% if non-KOMA class
  \IfFileExists{parskip.sty}{%
    \usepackage{parskip}
  }{% else
    \setlength{\parindent}{0pt}
    \setlength{\parskip}{6pt plus 2pt minus 1pt}}
}{% if KOMA class
  \KOMAoptions{parskip=half}}
\makeatother
\usepackage{xcolor}
\IfFileExists{xurl.sty}{\usepackage{xurl}}{} % add URL line breaks if available
\IfFileExists{bookmark.sty}{\usepackage{bookmark}}{\usepackage{hyperref}}
\hypersetup{
  pdftitle={Web Scrapping with R},
  pdfauthor={Author: Dickson Owuor (Ph.D.)},
  hidelinks,
  pdfcreator={LaTeX via pandoc}}
\urlstyle{same} % disable monospaced font for URLs
\usepackage[margin=1in]{geometry}
\usepackage{color}
\usepackage{fancyvrb}
\newcommand{\VerbBar}{|}
\newcommand{\VERB}{\Verb[commandchars=\\\{\}]}
\DefineVerbatimEnvironment{Highlighting}{Verbatim}{commandchars=\\\{\}}
% Add ',fontsize=\small' for more characters per line
\usepackage{framed}
\definecolor{shadecolor}{RGB}{248,248,248}
\newenvironment{Shaded}{\begin{snugshade}}{\end{snugshade}}
\newcommand{\AlertTok}[1]{\textcolor[rgb]{0.94,0.16,0.16}{#1}}
\newcommand{\AnnotationTok}[1]{\textcolor[rgb]{0.56,0.35,0.01}{\textbf{\textit{#1}}}}
\newcommand{\AttributeTok}[1]{\textcolor[rgb]{0.77,0.63,0.00}{#1}}
\newcommand{\BaseNTok}[1]{\textcolor[rgb]{0.00,0.00,0.81}{#1}}
\newcommand{\BuiltInTok}[1]{#1}
\newcommand{\CharTok}[1]{\textcolor[rgb]{0.31,0.60,0.02}{#1}}
\newcommand{\CommentTok}[1]{\textcolor[rgb]{0.56,0.35,0.01}{\textit{#1}}}
\newcommand{\CommentVarTok}[1]{\textcolor[rgb]{0.56,0.35,0.01}{\textbf{\textit{#1}}}}
\newcommand{\ConstantTok}[1]{\textcolor[rgb]{0.00,0.00,0.00}{#1}}
\newcommand{\ControlFlowTok}[1]{\textcolor[rgb]{0.13,0.29,0.53}{\textbf{#1}}}
\newcommand{\DataTypeTok}[1]{\textcolor[rgb]{0.13,0.29,0.53}{#1}}
\newcommand{\DecValTok}[1]{\textcolor[rgb]{0.00,0.00,0.81}{#1}}
\newcommand{\DocumentationTok}[1]{\textcolor[rgb]{0.56,0.35,0.01}{\textbf{\textit{#1}}}}
\newcommand{\ErrorTok}[1]{\textcolor[rgb]{0.64,0.00,0.00}{\textbf{#1}}}
\newcommand{\ExtensionTok}[1]{#1}
\newcommand{\FloatTok}[1]{\textcolor[rgb]{0.00,0.00,0.81}{#1}}
\newcommand{\FunctionTok}[1]{\textcolor[rgb]{0.00,0.00,0.00}{#1}}
\newcommand{\ImportTok}[1]{#1}
\newcommand{\InformationTok}[1]{\textcolor[rgb]{0.56,0.35,0.01}{\textbf{\textit{#1}}}}
\newcommand{\KeywordTok}[1]{\textcolor[rgb]{0.13,0.29,0.53}{\textbf{#1}}}
\newcommand{\NormalTok}[1]{#1}
\newcommand{\OperatorTok}[1]{\textcolor[rgb]{0.81,0.36,0.00}{\textbf{#1}}}
\newcommand{\OtherTok}[1]{\textcolor[rgb]{0.56,0.35,0.01}{#1}}
\newcommand{\PreprocessorTok}[1]{\textcolor[rgb]{0.56,0.35,0.01}{\textit{#1}}}
\newcommand{\RegionMarkerTok}[1]{#1}
\newcommand{\SpecialCharTok}[1]{\textcolor[rgb]{0.00,0.00,0.00}{#1}}
\newcommand{\SpecialStringTok}[1]{\textcolor[rgb]{0.31,0.60,0.02}{#1}}
\newcommand{\StringTok}[1]{\textcolor[rgb]{0.31,0.60,0.02}{#1}}
\newcommand{\VariableTok}[1]{\textcolor[rgb]{0.00,0.00,0.00}{#1}}
\newcommand{\VerbatimStringTok}[1]{\textcolor[rgb]{0.31,0.60,0.02}{#1}}
\newcommand{\WarningTok}[1]{\textcolor[rgb]{0.56,0.35,0.01}{\textbf{\textit{#1}}}}
\usepackage{graphicx}
\makeatletter
\def\maxwidth{\ifdim\Gin@nat@width>\linewidth\linewidth\else\Gin@nat@width\fi}
\def\maxheight{\ifdim\Gin@nat@height>\textheight\textheight\else\Gin@nat@height\fi}
\makeatother
% Scale images if necessary, so that they will not overflow the page
% margins by default, and it is still possible to overwrite the defaults
% using explicit options in \includegraphics[width, height, ...]{}
\setkeys{Gin}{width=\maxwidth,height=\maxheight,keepaspectratio}
% Set default figure placement to htbp
\makeatletter
\def\fps@figure{htbp}
\makeatother
\setlength{\emergencystretch}{3em} % prevent overfull lines
\providecommand{\tightlist}{%
  \setlength{\itemsep}{0pt}\setlength{\parskip}{0pt}}
\setcounter{secnumdepth}{-\maxdimen} % remove section numbering
\ifLuaTeX
  \usepackage{selnolig}  % disable illegal ligatures
\fi

\title{Web Scrapping with R}
\author{Author: Dickson Owuor (Ph.D.)}
\date{14/06/2022}

\begin{document}
\maketitle

{
\setcounter{tocdepth}{2}
\tableofcontents
}
\hypertarget{data-extraction}{%
\section{Data Extraction}\label{data-extraction}}

Data is increasingly becoming the lifeblood of the digital economy and
as most companies transit into online businesses, the importance of data
is increasing rapidly. For data to be useful, it has to be collected and
transformed into a form suitable for analysis.

the process of retrieving data from data sources for further data
processing or storage. In this tutorial, we use the \textbf{rvest} R
library to perform Web scrapping on a Wikipedia page. Web scrapping is
an example technique used for data extraction. The tutorial is adopted
from:

\begin{itemize}
\tightlist
\item
  \href{https://cran.r-project.org/web/packages/rvest/index.html}{CRAN
  rvest library}
\end{itemize}

\hypertarget{installation-of-libraries}{%
\section{Installation of libraries}\label{installation-of-libraries}}

Install and load the \textbf{rvest} package.

\begin{Shaded}
\begin{Highlighting}[]
\FunctionTok{install.packages}\NormalTok{(}\StringTok{"rvest"}\NormalTok{)}
\FunctionTok{library}\NormalTok{(rvest)}
\end{Highlighting}
\end{Shaded}

\hypertarget{usage-and-documentation}{%
\section{Usage and documentation}\label{usage-and-documentation}}

\begin{Shaded}
\begin{Highlighting}[]
\NormalTok{??rvest}
\end{Highlighting}
\end{Shaded}

\hypertarget{selecting-url-and-reading-html-content}{%
\section{Selecting URL and reading HTML
content}\label{selecting-url-and-reading-html-content}}

We begin the extraction process by specifying the URL of interest and
reading its contents.

\begin{Shaded}
\begin{Highlighting}[]
\NormalTok{wikiurl}\OtherTok{\textless{}{-}}\FunctionTok{read\_html}\NormalTok{(}\StringTok{"https://en.wikipedia.org/wiki/List\_of\_highest{-}grossing\_films"}\NormalTok{)}
\end{Highlighting}
\end{Shaded}

\hypertarget{filtering-data-of-interest}{%
\section{Filtering data of interest}\label{filtering-data-of-interest}}

In this tutorial, we are interested in scrapping \emph{`movie charts'}
tables from the URL.

\begin{Shaded}
\begin{Highlighting}[]
\CommentTok{\# Get the movie charts from wikipedia url using html\_table function}
\NormalTok{moviecharts }\OtherTok{\textless{}{-}} \FunctionTok{html\_table}\NormalTok{(wikiurl)}
\end{Highlighting}
\end{Shaded}

\hypertarget{selecting-data-of-interest}{%
\section{Selecting data of interest}\label{selecting-data-of-interest}}

There are multiple tables in the page, we specify the one we need and
store it in a data-frame.

\begin{Shaded}
\begin{Highlighting}[]
\NormalTok{moviecharts[[}\DecValTok{4}\NormalTok{]]}


\CommentTok{\#Put the movie chart into a data{-}frame}
\NormalTok{highestgrossfilms}\OtherTok{\textless{}{-}}\FunctionTok{data.frame}\NormalTok{(moviecharts[[}\DecValTok{4}\NormalTok{]])}
\NormalTok{highestgrossfilms}

\CommentTok{\#Numbers of rows and columns}
\FunctionTok{dim}\NormalTok{(highestgrossfilms)}
\end{Highlighting}
\end{Shaded}

\hypertarget{writing-interesting-data-to-file}{%
\section{Writing interesting data to
file}\label{writing-interesting-data-to-file}}

In order to write our data into a CSV file, check your working directory
and set it to your desired location. Then, we can write the data into
the CSV file.

\begin{Shaded}
\begin{Highlighting}[]
\FunctionTok{getwd}\NormalTok{()}
\FunctionTok{setwd}\NormalTok{(}\StringTok{"C:/Users/OwuorJnr/Desktop"}\NormalTok{) }\CommentTok{\# change this section to yours}

\FunctionTok{write.csv}\NormalTok{(highestgrossfilms,}\StringTok{"gross.csv"}\NormalTok{, }\AttributeTok{row.names =} \ConstantTok{TRUE}\NormalTok{)}
\end{Highlighting}
\end{Shaded}

\hypertarget{questions}{%
\section{Questions}\label{questions}}

\begin{enumerate}
\def\labelenumi{\arabic{enumi}.}
\tightlist
\item
  Which data output methods have been used in this exercise?
\item
  Is the data selected and saved in the CSV file, structure,
  unstructured or semi-structured? Explain your answer.
\item
  What type of logical extraction has been applied in this exercise?
  Explain your answer.
\item
  What type of physical extraction has been applied in this exercise?
  Explain your answer.
\item
  Briefly describe the ETL tasks that have been implemented in this
  exerxise. Which ones are missing?
\end{enumerate}

\hypertarget{exercise}{%
\section{Exercise}\label{exercise}}

This exercise is an assignment and you are required to:

\begin{enumerate}
\def\labelenumi{\alph{enumi}.}
\item
  Identify any Web page of your interest and perform all the steps above
  to scrap data of interest to you, and
\item
  store it into a data-frame and write it into a CSV file.
\end{enumerate}

\end{document}
